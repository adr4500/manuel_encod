\section{Presets}
\label{sec:presets}

Un preset est un ensemble de paramètres de fixtures qui peut être sauvegardé et modifié.

\subsection{Pourquoi utiliser des presets ?}
\label{subsec:presets_pourquoi}

Lorsqu'on prépare un show en avance, on n'a souvent pas accès aux vrais projecteurs, et ils ne sont de toute façon pas placés.
Il est donc difficile de régler les paramètres des fixtures sans les voir en action.
\newline
Pour pallier à ce problème, on peut créer des presets qui semblent convenir sur la vue 3D, mais qui sont ammenés à être modifiés une fois les projecteurs en place.
\newline
Ainsi, on peut utiliser et les valeurs des presets partout dans la programmation, il suffira de modifier uniquement les presets et toute la programmation s'adaptera.

\subsection{Créer un preset}
\label{subsec:presets_creer}

Pour créer un preset, selectionnez les fixtures que vous voulez inclure dans le preset,
ouvrez la fenetre \textit{Presets} en haut a droite, en mode \textit{Link to programmer}.
\newline
Ensuite, choisissez la catégorie de paramères que vous voulez enregistrer (\textit{Intensity}, \textit{Position}, \textit{Color}, etc.).
Ajustez les valeurs des paramètres à l'aide des encodeurs. Les valeurs en rouge sont celles qui vont être enregistrées.
\newline
Pour enregistrer le preset, appuyez sur \textit{Rec} et cliquez sur une case vide de la fenetre \textit{Presets}.
\newline
Appuyez sur la touche \textit{Clear} plusieurs fois pour reinitialiser l'état des fixtures, et recommencez pour les autres presets et fixtures.
\newline
\newline
\textbf{Attention :} Ne creez deux presets differents pour la même valeur sur deux types de fixtures différents.
Par exemple, ne créer pas un preset "Modena Rouge" et un preset "Dino Rouge", mais faites un seul preset "Rouge" qui s'applique à toutes les fixtures.
\newline
Si vous voulez tout de même enregistrer fixture par fixture, vous pouvez fusionner ce que vous souhaitez enregistrer avec ce qui a déjà été enregistré : appuyez sur \textit{Rec} et selectionnez le preset à fusionner.
Une fenêtre apparaitra, choisissez \textit{Merge} pour fusionner les valeurs.
\newline
\newline

\subsection{Quoi enregistrer ?}
\label{subsec:presets_quoi}

\subsubsection{Intensity}
\label{subsubsec:presets_intensity}
Enregistrez les valeurs de Dimmer 0\%, 50\% et 100\%.
Enregistrez plusieurs types de Shutter : Open, Strobe Fast, Strobe Slow, Closed.
\newline
\newline
Quand vous enregistrez un Dimmer, n'enregistrez pas de Shutter, et vice versa.

\subsubsection{Position}
\label{subsubsec:presets_position}
Si vous ne disposez pas de visualisateur, enregistrez à chaque fois une position 0° pan et tilt. Vous pourrez les modifier plus tard.
\newline
Enregistrez des positions du type : Divergent, Centre, Convergent, Croisé 1, Croisé 2 et toute autre position qui peut vous sembler interessante.
\newline
\newline
Pour créer rapidement ces positions, vous pouvez utiliser la fonction \textit{Fan}.
Selectionnez un groupe LTR, mettez le tilt vers le haut (public ou plus haut),
appuyez sur \textit{Fan} sur le clavier et modifiez le pan. Vous êtes en train
de créer une position de type Divergent ou Convergent.
\newline
Si vous selectionnez un groupe Sym, cela créera des positions croisées.
\newline
\newline
\textbf{Attention :} N'enregistrez pas dans les presets de position d'autres attributs comme "P/T Speed".
Si vous avez malencontreusement modifié une valeur que vous ne souhaitez pas enregistrer (et qui apparait donc en rouge),
cliquez sur la valeur, appuyez sur la touche \textit{Set} et choisissez \textit{Off} sur l'ecran.

\subsubsection{Color}
\label{subsubsec:presets_color}

Ici, vous allez créer un ensemble de couleurs de base : Rouge, Orange, Jaune, Vert, Cyan, Bleu, Violet, Magenta et Blanc.
\newline
Ici, vous allez vouloir enregistrer TOUS les attributs de couleur en même temps, à chaque fois. Toutes les roues, toutes les LEDs, etc.

\subsubsection{Gobo/Beam}
\label{subsubsec:presets_gobo_beam}

Du fait que les gobos et les beams sont souvent liés, je vous recommande de les enregistrer ensemble.
\newline
\newline
Pour cela, mettez la page \textit{Presets} en mode \textit{All}.
Enregistrez les ensembles de gobo, prisme, zoom ensemble. N'oubliez jamais d'enregistrer l'attribut de Focus dans les presets de ce type.

\subsubsection{Autres}
\label{subsubsec:presets_autres}

De manière analogue, vous pouvez enregistrer d'autres presets comme des Macros, des commandes Control (Lamp On par exemple).

\subsection{Faire un preset la lumière éteinte ?}
\label{subsec:presets_lumiere_eteinte}

Si vous n'avez pas de visualisateur, la question ne se pose pas, vous ferez tout à l'aveugle de toute façon.
\newline
Cependant, si vous avez un visualisateur, ou les lumières réelles devant vous, il serait bien de pouvoir voir ce que vous faites.
Pour cela, créez d'abord les presets d'intensité. Ensuite, réalisez ce qui est indiqué dans la section \ref{subsec:fader_full}.
A partir de là, vous pourrez contrôler l'intensité des fixtures avec les faders sans risquer d'enregistrer de valeur de dimmer.
Vous pouvez à présent continuer de créer vos autres presets.
\newline
\newline
Pro tip : Il est possible de donner des icônes aux presets. Pour cela, appuyez sur \textit{Edit} et cliquez sur le preset que vous voulez modifier. Vous pouvez choisir une icône ou une couleur parmi celles proposées.