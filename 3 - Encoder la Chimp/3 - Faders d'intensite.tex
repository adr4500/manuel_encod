\section{Faders d'intensité}
\label{sec:faders_intensite}

À partir de maintenant, nous allons enregistrer ce qu'on appelle des \textit{Cues}.
Un Cue est un ensemble de valeurs de projecteurs à un instant donné.
\newline
\newline
\textbf{Attention :} Il ne faut PAS enregistrer de valeurs brutes dans les Cues,
il faut toujours enregistrer une valeur de preset. Si vous n'avez pas le preset
que vous voulez, créez-le.

\subsection{Fader Full}
\label{subsec:fader_full}

L'idée ici est d'associer chaque fader au contrôle de l'intensité d'un groupe de projecteurs.
Généralement, on groupe les projecteurs par type.
\newline
\newline
Pour chaque groupe, selectionnez le dans la banque de groupes, selectionnez le preset "Dimmer 100\%" et appuyez sur la touche \textit{Rec} du clavier.
Cliquez sur le fader que vous voulez associer au groupe et donnez lui le nom du groupe.
\newline
Donner aussi une couleur au fader pour le retrouver plus facilement (voir section \ref{subsec:prep_colorer}).
\newline
\newline
Appuyez sur le bouton \textit{Edit} du clavier et appuyez ensuite sur le fader.
\newline
\newline
Dans l'onglet \textit{Edit Fader}, assignez le Pause button à l'action \textit{Back}.
\newline
Dans l'onglet \textit{Cuelist View}, double cliquez sur le nom de la Cue et appellez la "Full".
Reglez le "In Fade" sur 0s
\newline
Dans l'onglet \textit{Cuelist Settings} :
\begin{itemize}{}
    \item Réglez le "Tracking" sur Disabled
    \item Réglez le "Release Time" sur 0s
    \item Réglez le "Effect Speed Master" sur FX Speed Master 1
    \item Activez le "Dimmer HTP"
    \item Gardez les autres paramètres par défaut
\end{itemize}
Faites la même chose pour chaque groupe. N'oubliez pas de \textit{Clear} entre chaque groupe.
\newline
\newline
\textbf{Explication}
\newline
\newline
Regler le pause button à l'action Back permet de librement passer d'une cue à l'autre quand on en aura plusieurs enregistrées sur un seul fader.
\newline
Le In Fade est la durée avec laquelle la Cue va s'activer. Généralement, pour des cues d'intensité, on met 0s.
\newline
Le Tracking indique si la Cuelist retient les valeurs des Cues précédentes.
En le desactivant, on s'assure que chaque Cue de la Cuelist est indépendante, et on ne risque pas de mélanger tous les effets involontairement.
Il est interessant de l'activer dans le cas de conduite lumière comme un Timecode par exemple (voir section \ref{sec:timecode}).
\newline
Le Release Time est le temps que mettra la Cue à s'éteindre. Ici, on veut que les Cues s'éteignent instantanément.
\newline
L'Effect Speed Master est le fader qui va contrôler la vitesse des effets. Il sera utile pour les effets de type Chaser qu'on verra en section \ref{subsec:exec_effets}.
\newline
HTP signifie Highest Takes Precedence. Cela signifie que si deux Cues d'intensité sont actives en même temps, la valeur la plus haute des deux sera prise en compte.
C'est en opposition à LTP (Latest Takes Precedence) où c'est la dernière valeur executée qui est prise en compte.

\subsection{Fader Strobe}
\label{subsec:fader_strobe}

Pour chaque groupe, selectionnez le et selectionnez le preset "Dimmer 100\%" ainsi qu'un preset de Strobe de votre choix.
Enregistrez ça sur le fader déjà utilisé pour CE groupe. La console vous demandera si vous souhaitez \textit{Remove}, \textit{Replace}, \textit{Merge} ou \textit{Append}.
Choisissez \textit{Append} et entrez un nom logique pour la cue, comme "Strobe".
\newline
Pensez à modifier la valeur de In Fade de la Cue à 0s (voir section précédente).
\newline
\newline
Faites de même pour tous les groupes.

\subsection{Fader Chaser}
\label{subsec:fader_chaser}

Nous verrons les effets un petit peu plus tard, mais garder en tête que nous ajouterons une ou plusieurs Cues pour chaque groupe pour les effets de type Chaser.
