\section{Les exécuteurs}
\label{sec:executeurs}

Les executeurs sont des boutons (physiques ou virtuels) qui permettent de lancer divers éléments de programmation.

\subsection{Couleurs}
\label{subsec:exec_couleurs}

La première étape va être de créer des exécuteurs pour les couleurs de vos projecteurs.
\newline
Il y a deux façon de faire : soit vous créez un executeur par couleur et par groupe de projecteurs, soit vous créez des ambiances sur toutes vos lumières.
\newline
\newline
Dans les deux cas, le principe de création est le même : sélectionnez les fixtures que vous voulez inclure dans l'executeur et donnez leur une couleur grâce aux presets. Enregistrez ensuite l'executeur en appuyant sur \textit{Rec} et en cliquant sur une case vide de la fenêtre \textit{Executors} (sur la page 1 de cette fenêtre).
\newline
\newline
Il ne faut pas oublier de créer un executeur "Blanc" qui sera toujours activé, et qui contient toutes les fixtures en blanc. C'est important car la couleur par défaut
de certains projecteur peut ne pas être blanc. Par exemple, pour les projecteurs LED RGBW, la couleur par défaut (tout à 100\%) est un peu violet.
\newline
Cela ne signifie pas que toutes les lumières seront toujours blanches, mais si aucune autre couleur n'est selectionnée, les lumières seront blanches.
\newline
\newline
N'hesitez pas, particulièrement pour ces execteurs, de colorer les boutons d'executeurs (voir section \ref{subsec:prep_colorer}).
\newline
\newline
Les paramètres par defaut de la cuelist conviennent généralement bien, mais il peut être interessant d'ajuster le In Fade au besoin de votre spectacle.

\subsection{Positions}
\label{subsec:exec_positions}

Les executeurs de positions suivent le même principe que les couleurs. La seule
difference est qu'il faut ajouter le Master Fade 1 dans les paramètres de la cuelist.

\subsection{Gobo/Beam}
\label{subsec:exec_gobo_beam}

Les executeurs de Gobo/Beam suivent le même principe que les couleurs.
\newline
\newline
\textbf{IMPORTANT :} À partir de cette étape, il est important de visualiser ce que font les lumières (en vrai ou sur un visualiseur).
Si vous encodez à l'aveugle car le projecteur n'est pas disponible dans la librairie de votre visualisateur, c'est maintenant qu'il faut effectuer la procédure de morphing (voir section \ref{subsec:fixture_morph}).

\subsection{Effets}
\label{subsec:exec_effets}

Il est enfin temps d'aborder les effets.
\newline
Il s'agit de rendre dynamique certains paramètres dans le temps. Cela peut être un effet de Chaser (allumer des lumières séquentiellement), de mouvement, de couleur, etc.
\newline
\newline
Pour créer un effet, il faut avant tout selectionner les fixtures sur lequel il va s'appliquer.
Ensuite, il faut se rendre dans la partie \textit{Effects} du programmer (en page 2 en bas à droite),
et cliquer sur \textit{Add FX}. Vous avez alors toute une librairie d'effets prédéfinis à votre disposition.
Essayez les, modifiez les differents paramètres afin de trouver ce qui vous convient le mieux. Si vous
souhaitez en savoir plus sur la création d'effet et le système de génération des effets, le manuel de la Chimp est plutôt bien fait à ce sujet.
Une fois l'effet désiré trouvé, enregistrez le dans un executeur en appuyant sur \textit{Rec} et en cliquant sur une case vide de la fenêtre \textit{Executors} (sur une autre page que les couleurs).
Pensez bien à ajouter le FX Speed Master 2 dans les paramètres de la cuelist ainsi crée.
\newline
\newline
C'est aussi maintenant que vous allez pouvoir générer des effets de Chaser à mettre dans les faders. Mettez un chaser sur les Dinos dans
le fader des Dinos par exemple.

\subsection{Cuelist Chase}
\label{subsec:exec_cuelist_chase}

Pour l'instant, nous n'avons utilisé les cuelists que dans leur mode par défaut.
Mais elles ont aussi un mode "Chase" qui permet d'enchainer plusieurs cues en boucle.
C'est utile quand vous voulez faire une alternance rapide entre plusieurs états.
\newline
Si vous créez un executeur de cuelist chase, il faut ajouter le Speed Master 1 dans les paramètres de la cuelist.

\subsection{Executeurs physiques}
\label{subsec:exec_physiques}

Dans les executeurs physiques (10 boutons à droite des faders), je vous recommande de mettre des effets très dynamiques,
comme un Strobe Blackout, un flash de couleur ou une alternance de dimmer.
\newline
Au besoin, vous pouvez assigner jusque 10 pages de 10 executeurs physiques.

\subsection{C'est prêt !}
\label{subsec:exec_pret}

Votre show est maintenant intégralement programmé. Voici quelques clés pour l'utiliser efficacement.
Le FX Speed Master 1 (premier fader à droite) permet de régler la vitesse des effets d'intensité qui sont sur les faders.
Le FX Speed Master 2 (deuxième fader à droite) permet de régler la vitesse des autres effets présents dans les executeurs.
Le Speed Master 1 (troisième fader à droite) permet de régler la vitesse des cuelists en mode chase. On peut utiliser le bouton en "Tap BPM" pour synchroniser la vitesse de l'effet à la musique.
Le Fade Master (quatrième fader à droite) permet de régler la durée des transitions entre les positions dans les executeurs.
\newline
\newline
Pour passer d'un effet d'intensité à l'autre, utilisez les boutons au dessus des faders. Le bouton en dessous sert à allumer le groupe instantanément.
\newline
\newline
Pensez bien a toujours garder activé l'executeur "Blanc" pour éviter les couleurs moches.
\newline
\newline
Enfin, entrainez vous à utiliser votre encod sur des musiques pour vous y habituer.
