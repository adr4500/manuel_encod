\chapter{Introduction}
\label{chap:introduction}

Bonjour cher.e lecteur.trice, et bienvenue dans ce guide de l'encodage lumière.
Ce manuel s'adresse à toute personne souhaitant programmer un spectacle sur la console Infinity Chimp 300.
Quel que soit votre niveau de connaissance, après lecture de ce document, vous devriez avoir toutes les clés pour préparer au mieux la console pour un spectacle.
\newline
\newline
Dans ce guide, nous aborderons les bases de la programmation lumière, toute console confondue.
Nous parlerons des protocoles qui permettent de communiquer avec les projecteurs, du lexique associé aux elements programmables et des bonnes pratiques pour préparer un spectacle.
\newline
Nous aborderons ensuite en détail les étapes à suivre pour programmer sur la console.
\newline
Nous verrons aussi les differents outils à notre disposition pour adapter une programmation déjà réalisée en cas de changement quelconque.
\newline
Enfin, nous présenterons quelques outils utiles dans des situations spécifiques ou moins courantes.
\newline
\newline
Il est important de noter que ce guide est particulièrement destiné aux membres de l'association Décibels de l'UTC,
et ainsi, les exemples seront souvent en lien avec le matériel et les activités de l'association.
Cependant, les informations contenues dans ce guide sont générales et peuvent être utilisées pour tout type de spectacle.
\newline
\newline
Bonne lecture !
